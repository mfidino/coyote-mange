% Options for packages loaded elsewhere
\PassOptionsToPackage{unicode}{hyperref}
\PassOptionsToPackage{hyphens}{url}
%
\documentclass[
]{article}
\usepackage{lmodern}
\usepackage{amssymb,amsmath}
\usepackage{ifxetex,ifluatex}
\ifnum 0\ifxetex 1\fi\ifluatex 1\fi=0 % if pdftex
  \usepackage[T1]{fontenc}
  \usepackage[utf8]{inputenc}
  \usepackage{textcomp} % provide euro and other symbols
\else % if luatex or xetex
  \usepackage{unicode-math}
  \defaultfontfeatures{Scale=MatchLowercase}
  \defaultfontfeatures[\rmfamily]{Ligatures=TeX,Scale=1}
\fi
% Use upquote if available, for straight quotes in verbatim environments
\IfFileExists{upquote.sty}{\usepackage{upquote}}{}
\IfFileExists{microtype.sty}{% use microtype if available
  \usepackage[]{microtype}
  \UseMicrotypeSet[protrusion]{basicmath} % disable protrusion for tt fonts
}{}
\makeatletter
\@ifundefined{KOMAClassName}{% if non-KOMA class
  \IfFileExists{parskip.sty}{%
    \usepackage{parskip}
  }{% else
    \setlength{\parindent}{0pt}
    \setlength{\parskip}{6pt plus 2pt minus 1pt}}
}{% if KOMA class
  \KOMAoptions{parskip=half}}
\makeatother
\usepackage{xcolor}
\IfFileExists{xurl.sty}{\usepackage{xurl}}{} % add URL line breaks if available
\IfFileExists{bookmark.sty}{\usepackage{bookmark}}{\usepackage{hyperref}}
\hypersetup{
  pdftitle={Simulated occupancy image analysis for Murray et al.},
  pdfauthor={Mason Fidino},
  hidelinks,
  pdfcreator={LaTeX via pandoc}}
\urlstyle{same} % disable monospaced font for URLs
\usepackage[margin=1in]{geometry}
\usepackage{color}
\usepackage{fancyvrb}
\newcommand{\VerbBar}{|}
\newcommand{\VERB}{\Verb[commandchars=\\\{\}]}
\DefineVerbatimEnvironment{Highlighting}{Verbatim}{commandchars=\\\{\}}
% Add ',fontsize=\small' for more characters per line
\usepackage{framed}
\definecolor{shadecolor}{RGB}{248,248,248}
\newenvironment{Shaded}{\begin{snugshade}}{\end{snugshade}}
\newcommand{\AlertTok}[1]{\textcolor[rgb]{0.94,0.16,0.16}{#1}}
\newcommand{\AnnotationTok}[1]{\textcolor[rgb]{0.56,0.35,0.01}{\textbf{\textit{#1}}}}
\newcommand{\AttributeTok}[1]{\textcolor[rgb]{0.77,0.63,0.00}{#1}}
\newcommand{\BaseNTok}[1]{\textcolor[rgb]{0.00,0.00,0.81}{#1}}
\newcommand{\BuiltInTok}[1]{#1}
\newcommand{\CharTok}[1]{\textcolor[rgb]{0.31,0.60,0.02}{#1}}
\newcommand{\CommentTok}[1]{\textcolor[rgb]{0.56,0.35,0.01}{\textit{#1}}}
\newcommand{\CommentVarTok}[1]{\textcolor[rgb]{0.56,0.35,0.01}{\textbf{\textit{#1}}}}
\newcommand{\ConstantTok}[1]{\textcolor[rgb]{0.00,0.00,0.00}{#1}}
\newcommand{\ControlFlowTok}[1]{\textcolor[rgb]{0.13,0.29,0.53}{\textbf{#1}}}
\newcommand{\DataTypeTok}[1]{\textcolor[rgb]{0.13,0.29,0.53}{#1}}
\newcommand{\DecValTok}[1]{\textcolor[rgb]{0.00,0.00,0.81}{#1}}
\newcommand{\DocumentationTok}[1]{\textcolor[rgb]{0.56,0.35,0.01}{\textbf{\textit{#1}}}}
\newcommand{\ErrorTok}[1]{\textcolor[rgb]{0.64,0.00,0.00}{\textbf{#1}}}
\newcommand{\ExtensionTok}[1]{#1}
\newcommand{\FloatTok}[1]{\textcolor[rgb]{0.00,0.00,0.81}{#1}}
\newcommand{\FunctionTok}[1]{\textcolor[rgb]{0.00,0.00,0.00}{#1}}
\newcommand{\ImportTok}[1]{#1}
\newcommand{\InformationTok}[1]{\textcolor[rgb]{0.56,0.35,0.01}{\textbf{\textit{#1}}}}
\newcommand{\KeywordTok}[1]{\textcolor[rgb]{0.13,0.29,0.53}{\textbf{#1}}}
\newcommand{\NormalTok}[1]{#1}
\newcommand{\OperatorTok}[1]{\textcolor[rgb]{0.81,0.36,0.00}{\textbf{#1}}}
\newcommand{\OtherTok}[1]{\textcolor[rgb]{0.56,0.35,0.01}{#1}}
\newcommand{\PreprocessorTok}[1]{\textcolor[rgb]{0.56,0.35,0.01}{\textit{#1}}}
\newcommand{\RegionMarkerTok}[1]{#1}
\newcommand{\SpecialCharTok}[1]{\textcolor[rgb]{0.00,0.00,0.00}{#1}}
\newcommand{\SpecialStringTok}[1]{\textcolor[rgb]{0.31,0.60,0.02}{#1}}
\newcommand{\StringTok}[1]{\textcolor[rgb]{0.31,0.60,0.02}{#1}}
\newcommand{\VariableTok}[1]{\textcolor[rgb]{0.00,0.00,0.00}{#1}}
\newcommand{\VerbatimStringTok}[1]{\textcolor[rgb]{0.31,0.60,0.02}{#1}}
\newcommand{\WarningTok}[1]{\textcolor[rgb]{0.56,0.35,0.01}{\textbf{\textit{#1}}}}
\usepackage{graphicx,grffile}
\makeatletter
\def\maxwidth{\ifdim\Gin@nat@width>\linewidth\linewidth\else\Gin@nat@width\fi}
\def\maxheight{\ifdim\Gin@nat@height>\textheight\textheight\else\Gin@nat@height\fi}
\makeatother
% Scale images if necessary, so that they will not overflow the page
% margins by default, and it is still possible to overwrite the defaults
% using explicit options in \includegraphics[width, height, ...]{}
\setkeys{Gin}{width=\maxwidth,height=\maxheight,keepaspectratio}
% Set default figure placement to htbp
\makeatletter
\def\fps@figure{htbp}
\makeatother
\setlength{\emergencystretch}{3em} % prevent overfull lines
\providecommand{\tightlist}{%
  \setlength{\itemsep}{0pt}\setlength{\parskip}{0pt}}
\setcounter{secnumdepth}{-\maxdimen} % remove section numbering
\usepackage{bm}

\title{Simulated occupancy image analysis for Murray et al.}
\author{Mason Fidino}
\date{6/26/2020}

\begin{document}
\maketitle

\hypertarget{setting-the-stage-the-simulation}{%
\subsection{Setting the stage the
simulation}\label{setting-the-stage-the-simulation}}

To keep things tractable we will simulate data for a single season at
\(n = 300\) sites. Finally, we assume that there a total of \(J = 4\)
repeat visits to each site within this single sampling season (e.g.,
four weeks of camera trapping).\\

\begin{Shaded}
\begin{Highlighting}[]
\CommentTok{# Set seed for reproducibility}
\KeywordTok{set.seed}\NormalTok{(}\OperatorTok{-}\DecValTok{565}\NormalTok{)}

\CommentTok{# The number of sites for this example}
\NormalTok{nsite <-}\StringTok{ }\DecValTok{300}

\CommentTok{# The number of visits to each site}
\NormalTok{j <-}\StringTok{ }\DecValTok{4}
\end{Highlighting}
\end{Shaded}

\hypertarget{the-occupancy-model}{%
\subsection{The Occupancy model}\label{the-occupancy-model}}

This is for the first part of the model. In our own analysis, this part
of the analysis estimates the proability of coyote occupancy, mangy or
otherwise, as well as the probability of detecting coyote, mangy or
otherwise, on one of the repeat visits assuming they are present. More
generally, the log-odds a species is detected at \(s\) in \(1,...,S\)
sites is \(logit(\psi_s) = \bm{a^T x_s}\), where \(\bm{a}\) is a vector
of parameters and \(\bm{X}\) is row \(s\) of a conformable matrix of
covariates where the first column is all \(1's\) to account for the
intercept. For this simulation we will use one covariate on site
occupancy. Thus, we will need to specify an intercept (\(a_0\)) and
slope term (\(a_1\)) to estimate from the simulated data.

\[logit(\psi_s) = \bm{a^T x_s} = a_0 x_{s,1} + a_1 x_{s,2} = 0.5 - 1 x_s\]\\
With our environmental covariate (\(\bm{x}\)) and these logit-scaled
parameters we can simulate the occupancy status of this species at the
\(S\) sites.

\begin{Shaded}
\begin{Highlighting}[]
\CommentTok{# An environmental covariate for occupancy}
\NormalTok{x1 <-}\StringTok{ }\KeywordTok{rnorm}\NormalTok{(nsite)}

\CommentTok{# Occupancy  logit-linear predictor}
\NormalTok{z_det <-}\StringTok{ }\FloatTok{0.5} \OperatorTok{-}\StringTok{ }\DecValTok{1} \OperatorTok{*}\StringTok{ }\NormalTok{x1 }

\CommentTok{# Probability of occupancy}
\NormalTok{z_prob <-}\StringTok{ }\KeywordTok{plogis}\NormalTok{(}
\NormalTok{  z_det}
\NormalTok{)}

\CommentTok{# True occupancy status of species}
\NormalTok{z <-}\StringTok{ }\KeywordTok{rbinom}\NormalTok{(}
\NormalTok{  nsite,}
  \DecValTok{1}\NormalTok{,}
\NormalTok{  z_prob}
\NormalTok{)}
\end{Highlighting}
\end{Shaded}

If you have been following along then this species should be at
\texttt{sum(z)\ ==\ 180} sites.\\
We do not perfectly observe the occupancy status of this species. We
will let the log-odds we detect this species to vary along one
covariate.

\[logit(\rho_s) = \bm{b^T x_s} = b_0 x_{s,1} + b_1 x_{s,2} = -0.5 + 0.5 x_s\]\\

\begin{Shaded}
\begin{Highlighting}[]
\CommentTok{# A different covariate for detection}
\NormalTok{x2 <-}\StringTok{ }\KeywordTok{rnorm}\NormalTok{(}
\NormalTok{  nsite}
\NormalTok{)}

\CommentTok{# Logit-linear predictor}
\NormalTok{y_det <-}\StringTok{ }\FloatTok{-0.5} \OperatorTok{+}\StringTok{ }\FloatTok{0.5} \OperatorTok{*}\StringTok{ }\NormalTok{x2}

\CommentTok{# The probability of detection given presence}
\NormalTok{y_prob <-}\StringTok{ }\KeywordTok{plogis}\NormalTok{(}
\NormalTok{  y_det}
\NormalTok{)}

\CommentTok{# The observed data }
\CommentTok{#  Detection probability is 0 if species is not present}
\NormalTok{y <-}\StringTok{ }\KeywordTok{rbinom}\NormalTok{(}
\NormalTok{  nsite,}
\NormalTok{  j,}
\NormalTok{  y_prob }\OperatorTok{*}\StringTok{ }\NormalTok{z}
\NormalTok{)}
\end{Highlighting}
\end{Shaded}

Thus, in this simulation we failed to detect this species at
\texttt{sum(z)\ -\ sum(y\textgreater{}0)\ ==\ 29} sites.\\
That is all there is to the first part of the model. It's just a
standard occupancy model and could be traded out for literally any other
style of occupancy model you may be interested in using.

\hypertarget{the-conditional-by-image-model}{%
\subsection{The conditional by-image
model}\label{the-conditional-by-image-model}}

In our analysis this is the conditional mange model, as that is the
additional state we are interested in estimating. More generally, all
that is needed is for there to be an additional state that can be
observed on the images of a target species. The latent-state part of
this model is very similar to the latent-state of the occupancy model
except it can only happen if the species is present at the site. We
cannot view a `coyote with mange' if `coyote, mangy or otherwise' are
not present. The log-odds of this latent state is\\
\[logit(\omega_s) = \bm{c^T x_s} = c_0 x_{s,1} + c_1 x_{s,2} = -0.5 + 0.5 x_s\]\\

\begin{Shaded}
\begin{Highlighting}[]
\CommentTok{# Mange logit-linear predictor.}
\CommentTok{#  We will use the same covariate as we did with the latent}
\CommentTok{#  occupancy state.}
\NormalTok{w_det <-}\StringTok{ }\FloatTok{-0.5} \OperatorTok{+}\StringTok{ }\FloatTok{0.5} \OperatorTok{*}\StringTok{ }\NormalTok{x1 }

\CommentTok{# the probability a coyote is mangy at the site}
\NormalTok{w_prob <-}\StringTok{ }\KeywordTok{plogis}\NormalTok{(}
\NormalTok{  w_det}
\NormalTok{) }

\CommentTok{# The mange status of coyote across the sites}
\NormalTok{w <-}\StringTok{ }\KeywordTok{rbinom}\NormalTok{(}
\NormalTok{  nsite,}
  \DecValTok{1}\NormalTok{,}
\NormalTok{  w_prob }\OperatorTok{*}\StringTok{ }\NormalTok{z}
\NormalTok{)}
\end{Highlighting}
\end{Shaded}

The observation model here substantially varies from a standard
occupancy model because each site has a different number of images. We
know that every site we detected the species will have at least one
image. As we are going to iterate through all of the images we simply
need a way to link an image to the appropriate site within the model.
Likewise, for this simulation we will generate a different number of
images per site.\\

\begin{Shaded}
\begin{Highlighting}[]
\CommentTok{# Step 1. figure out where we collected photos}
\NormalTok{sites_with_photos <-}\StringTok{ }\KeywordTok{which}\NormalTok{(}
\NormalTok{  y}\OperatorTok{>}\DecValTok{0}
\NormalTok{)}

\CommentTok{# Step 2. simulate number of photos per site}
\NormalTok{photos_per_site <-}\StringTok{ }\KeywordTok{sample}\NormalTok{(}
  \DecValTok{1}\OperatorTok{:}\DecValTok{30}\NormalTok{,}
  \KeywordTok{length}\NormalTok{(sites_with_photos),}
  \DataTypeTok{replace =} \OtherTok{TRUE}
\NormalTok{)}

\CommentTok{# This is the total number of images}
\NormalTok{n_photos <-}\StringTok{ }\KeywordTok{sum}\NormalTok{(}
\NormalTok{  photos_per_site}
\NormalTok{)}
\end{Highlighting}
\end{Shaded}

Now that we have the images, we are going to introduce uncertainty in
our ability to detect a mangy coyote in an image given that mangy coyote
are present at a site. Let the log-odds we detect a mangy coyote in
image \(i\) be:\\
\[logit(\gamma_i) = \bm{d^T x_i} = d_0 x_{i,1} + d_1 x_{i,2} + d_2 x_{i,3}= -0.5 + 0.25 x_{i,2} + 0.7 x_{i,3}\]
Now, we need \_to come up with a way to link images to sites. We can use
a vector of length \texttt{n\_photos} where each element is an integer
that represents what site image \(i\) belongs to.

\begin{Shaded}
\begin{Highlighting}[]
\CommentTok{# This is a data.frame where each row has info on the sites}
\CommentTok{#  with images and the number of images at that site.}
\NormalTok{my_sites <-}\StringTok{ }\KeywordTok{data.frame}\NormalTok{(}
  \DataTypeTok{sites =}\NormalTok{ sites_with_photos,}
  \DataTypeTok{count =}\NormalTok{ photos_per_site}
\NormalTok{)}

\CommentTok{# This is the site index. It is of length n_photos. Each}
\CommentTok{#  element represents the site an image belongs to.}
\NormalTok{site_idx <-}\StringTok{ }\KeywordTok{rep}\NormalTok{(}
\NormalTok{  my_sites}\OperatorTok{$}\NormalTok{sites,}
\NormalTok{  my_sites}\OperatorTok{$}\NormalTok{count}
\NormalTok{)}
\end{Highlighting}
\end{Shaded}

The vector \texttt{site\_idx} can be used to link the images to sites.\\

\begin{Shaded}
\begin{Highlighting}[]
\CommentTok{# Covariates, one binary, one continuous}
\NormalTok{x3 <-}\StringTok{ }\KeywordTok{rnorm}\NormalTok{(}
\NormalTok{  n_photos}
\NormalTok{)}

\NormalTok{x4 <-}\StringTok{ }\KeywordTok{rbinom}\NormalTok{(}
\NormalTok{  n_photos,}
  \DecValTok{1}\NormalTok{,}
  \FloatTok{0.3}
\NormalTok{)}

\CommentTok{# Logit-linear predictor for detecting mange given presence}
\NormalTok{g_det <-}\StringTok{ }\FloatTok{-0.5} \OperatorTok{+}\StringTok{ }\FloatTok{0.25} \OperatorTok{*}\StringTok{ }\NormalTok{x3 }\OperatorTok{+}\StringTok{ }\FloatTok{0.7} \OperatorTok{*}\StringTok{ }\NormalTok{x4}

\CommentTok{# The probability of detecting mange in an image given presence}
\NormalTok{g_prob <-}\StringTok{ }\KeywordTok{plogis}\NormalTok{(}
\NormalTok{  g_det}
\NormalTok{)}

\CommentTok{# A binary vector that determines which sites we have observed}
\CommentTok{#  the species. We use this and the 'w' vector to simulate}
\CommentTok{#  our observed data. Basically we need a photo at the site}
\CommentTok{#  and mangy coyote must also be at the site.}
\NormalTok{y_observed <-}\StringTok{ }\KeywordTok{as.numeric}\NormalTok{(}
\NormalTok{  y}\OperatorTok{>}\DecValTok{0}
\NormalTok{)}

\CommentTok{# simulate the observed mange data}
\NormalTok{g <-}\StringTok{ }\KeywordTok{rbinom}\NormalTok{(}
\NormalTok{  n_photos,}
  \DecValTok{1}\NormalTok{,}
\NormalTok{  g_prob }\OperatorTok{*}\StringTok{ }\KeywordTok{as.numeric}\NormalTok{(}
\NormalTok{    y_observed[site_idx] }\OperatorTok{*}\StringTok{ }\NormalTok{w[site_idx]}
\NormalTok{  )}
\NormalTok{)}
\end{Highlighting}
\end{Shaded}

\#\# Fitting the model to simulated data The JAGS model we are using is
\texttt{"./jags\_script/conditional\_model\_single\_season.R"}. The code
for this is

\begin{Shaded}
\begin{Highlighting}[]
\NormalTok{model\{}
  \CommentTok{# priors}
  \ControlFlowTok{for}\NormalTok{(pc }\ControlFlowTok{in} \DecValTok{1}\OperatorTok{:}\NormalTok{ncov_psi)\{}
\NormalTok{    psi[pc] }\OperatorTok{~}\StringTok{ }\KeywordTok{dlogis}\NormalTok{(}\DecValTok{0}\NormalTok{, }\DecValTok{1}\NormalTok{)}
\NormalTok{  \}}
  \ControlFlowTok{for}\NormalTok{(rc }\ControlFlowTok{in} \DecValTok{1}\OperatorTok{:}\NormalTok{ncov_rho)\{}
\NormalTok{    rho[rc] }\OperatorTok{~}\StringTok{ }\KeywordTok{dlogis}\NormalTok{(}\DecValTok{0}\NormalTok{,}\DecValTok{1}\NormalTok{)}
\NormalTok{  \}}
  \ControlFlowTok{for}\NormalTok{(oc }\ControlFlowTok{in} \DecValTok{1}\OperatorTok{:}\NormalTok{ncov_omega)\{}
\NormalTok{    omega[oc] }\OperatorTok{~}\StringTok{ }\KeywordTok{dlogis}\NormalTok{(}\DecValTok{0}\NormalTok{,}\DecValTok{1}\NormalTok{)}
\NormalTok{  \}}
  \ControlFlowTok{for}\NormalTok{(gc }\ControlFlowTok{in} \DecValTok{1}\OperatorTok{:}\NormalTok{ncov_gamma)\{}
\NormalTok{    gamma[gc] }\OperatorTok{~}\StringTok{ }\KeywordTok{dlogis}\NormalTok{(}\DecValTok{0}\NormalTok{,}\DecValTok{1}\NormalTok{)}
\NormalTok{  \}}
  \ControlFlowTok{for}\NormalTok{(site }\ControlFlowTok{in} \DecValTok{1}\OperatorTok{:}\NormalTok{nsite)\{}
    \KeywordTok{logit}\NormalTok{(psi_mu[site]) <-}\StringTok{ }\KeywordTok{inprod}\NormalTok{(psi, psi_cov[site,])}
\NormalTok{    z[site] }\OperatorTok{~}\StringTok{ }\KeywordTok{dbern}\NormalTok{(psi_mu[site])}
\NormalTok{  \}}
  \ControlFlowTok{for}\NormalTok{(site }\ControlFlowTok{in} \DecValTok{1}\OperatorTok{:}\NormalTok{nsite)\{}
    \KeywordTok{logit}\NormalTok{(rho_mu[site]) <-}\StringTok{ }\KeywordTok{inprod}\NormalTok{(rho, rho_cov[site,])}
\NormalTok{    y[site] }\OperatorTok{~}\StringTok{ }\KeywordTok{dbin}\NormalTok{(rho_mu[site] }\OperatorTok{*}\StringTok{ }\NormalTok{z[site], J[site])}
\NormalTok{  \}}
  \ControlFlowTok{for}\NormalTok{(site }\ControlFlowTok{in} \DecValTok{1}\OperatorTok{:}\NormalTok{nsite)\{}
    \KeywordTok{logit}\NormalTok{(ome_mu[site]) <-}\StringTok{ }\KeywordTok{inprod}\NormalTok{(omega, omega_cov[site,])}
\NormalTok{    x[site] }\OperatorTok{~}\StringTok{ }\KeywordTok{dbern}\NormalTok{(ome_mu[site] }\OperatorTok{*}\StringTok{ }\NormalTok{z[site])}
\NormalTok{  \}}
  \ControlFlowTok{for}\NormalTok{(photo }\ControlFlowTok{in} \DecValTok{1}\OperatorTok{:}\NormalTok{nphoto)\{}
    \KeywordTok{logit}\NormalTok{(gam_mu[photo]) <-}\StringTok{ }\KeywordTok{inprod}\NormalTok{(gamma, gamma_cov[photo,])}
\NormalTok{    q[photo] }\OperatorTok{~}\StringTok{ }\KeywordTok{dbern}\NormalTok{(gam_mu[photo] }\OperatorTok{*}\StringTok{ }\NormalTok{x[site_vec[photo]])}
\NormalTok{  \}}
  \CommentTok{# derived quantities}

\NormalTok{    n_coyote <-}\StringTok{ }\KeywordTok{sum}\NormalTok{(z)}
\NormalTok{    n_mange <-}\StringTok{  }\KeywordTok{sum}\NormalTok{(x)}

\NormalTok{\}}
\end{Highlighting}
\end{Shaded}

And here is how we fit the simulated data to the \texttt{JAGS} model
using the \texttt{run.jags} packages.

\begin{Shaded}
\begin{Highlighting}[]
\CommentTok{# This is the initial values for whether mange is at a site. It}
\CommentTok{#  equals 1 if we observed it and is NA otherwise.}
\NormalTok{x_guess <-}\StringTok{ }\KeywordTok{rep}\NormalTok{(}
  \OtherTok{NA}\NormalTok{, }
\NormalTok{  nsite}
\NormalTok{) }
\NormalTok{x_guess[}\KeywordTok{unique}\NormalTok{(site_idx[g }\OperatorTok{==}\StringTok{ }\DecValTok{1}\NormalTok{])] <-}\StringTok{ }\DecValTok{1}

\CommentTok{# The number of parameters we are estimating}
\NormalTok{ncov_psi <-}\StringTok{ }\DecValTok{2}
\NormalTok{ncov_rho <-}\StringTok{ }\DecValTok{2}
\NormalTok{ncov_omega <-}\StringTok{ }\DecValTok{2}
\NormalTok{ncov_gamma <-}\StringTok{ }\DecValTok{3}

\CommentTok{# put together the data list that we need for this analysis}
\NormalTok{data_list <-}\StringTok{ }\KeywordTok{list}\NormalTok{(}
  \DataTypeTok{y =}\NormalTok{ y, }
  \DataTypeTok{q =}\NormalTok{ g, }
  \DataTypeTok{psi_cov =} \KeywordTok{cbind}\NormalTok{(}\DecValTok{1}\NormalTok{, x1),}
  \DataTypeTok{rho_cov =} \KeywordTok{cbind}\NormalTok{(}\DecValTok{1}\NormalTok{, x2),}
  \DataTypeTok{omega_cov =} \KeywordTok{cbind}\NormalTok{(}\DecValTok{1}\NormalTok{, x1),}
  \DataTypeTok{gamma_cov =} \KeywordTok{cbind}\NormalTok{(}\DecValTok{1}\NormalTok{, x3, x4),}
  \DataTypeTok{nphoto =}\NormalTok{ n_photos,}
  \DataTypeTok{nsite =}\NormalTok{ nsite,}
  \DataTypeTok{site_vec =}\NormalTok{ site_idx,}
  \DataTypeTok{ncov_psi =}\NormalTok{ ncov_psi,}
  \DataTypeTok{ncov_rho =}\NormalTok{ ncov_rho,}
  \DataTypeTok{ncov_omega =}\NormalTok{ ncov_omega,}
  \DataTypeTok{ncov_gamma =}\NormalTok{ ncov_gamma,}
  \DataTypeTok{J =} \KeywordTok{rep}\NormalTok{(j,nsite)}
\NormalTok{)}


\CommentTok{# generate initial values for the analysis}
\NormalTok{inits_simulated <-}\StringTok{ }\ControlFlowTok{function}\NormalTok{(chain)\{}
\NormalTok{  gen_list <-}\StringTok{ }\ControlFlowTok{function}\NormalTok{(}\DataTypeTok{chain =}\NormalTok{ chain)\{}
    \KeywordTok{list}\NormalTok{( }
      \DataTypeTok{z =} \KeywordTok{as.numeric}\NormalTok{(y_observed),}
      \DataTypeTok{x =}\NormalTok{ x_guess,}
      \DataTypeTok{psi =} \KeywordTok{rnorm}\NormalTok{(ncov_psi),}
      \DataTypeTok{rho =} \KeywordTok{rnorm}\NormalTok{(ncov_rho),}
      \DataTypeTok{omega =} \KeywordTok{rnorm}\NormalTok{(ncov_omega),}
      \DataTypeTok{gamma =} \KeywordTok{rnorm}\NormalTok{(ncov_gamma),}
      \DataTypeTok{.RNG.name =} \ControlFlowTok{switch}\NormalTok{(chain,}
                         \StringTok{"1"}\NormalTok{ =}\StringTok{ "base::Wichmann-Hill"}\NormalTok{,}
                         \StringTok{"2"}\NormalTok{ =}\StringTok{ "base::Marsaglia-Multicarry"}\NormalTok{,}
                         \StringTok{"3"}\NormalTok{ =}\StringTok{ "base::Super-Duper"}\NormalTok{,}
                         \StringTok{"4"}\NormalTok{ =}\StringTok{ "base::Mersenne-Twister"}\NormalTok{,}
                         \StringTok{"5"}\NormalTok{ =}\StringTok{ "base::Wichmann-Hill"}\NormalTok{,}
                         \StringTok{"6"}\NormalTok{ =}\StringTok{ "base::Marsaglia-Multicarry"}\NormalTok{,}
                         \StringTok{"7"}\NormalTok{ =}\StringTok{ "base::Super-Duper"}\NormalTok{,}
                         \StringTok{"8"}\NormalTok{ =}\StringTok{ "base::Mersenne-Twister"}\NormalTok{),}
      \DataTypeTok{.RNG.seed =} \KeywordTok{sample}\NormalTok{(}\DecValTok{1}\OperatorTok{:}\FloatTok{1e+06}\NormalTok{, }\DecValTok{1}\NormalTok{)}
\NormalTok{    )}
\NormalTok{  \}}
  \KeywordTok{return}\NormalTok{(}\ControlFlowTok{switch}\NormalTok{(chain,           }
                \StringTok{"1"}\NormalTok{ =}\StringTok{ }\KeywordTok{gen_list}\NormalTok{(chain),}
                \StringTok{"2"}\NormalTok{ =}\StringTok{ }\KeywordTok{gen_list}\NormalTok{(chain),}
                \StringTok{"3"}\NormalTok{ =}\StringTok{ }\KeywordTok{gen_list}\NormalTok{(chain),}
                \StringTok{"4"}\NormalTok{ =}\StringTok{ }\KeywordTok{gen_list}\NormalTok{(chain),}
                \StringTok{"5"}\NormalTok{ =}\StringTok{ }\KeywordTok{gen_list}\NormalTok{(chain),}
                \StringTok{"6"}\NormalTok{ =}\StringTok{ }\KeywordTok{gen_list}\NormalTok{(chain),}
                \StringTok{"7"}\NormalTok{ =}\StringTok{ }\KeywordTok{gen_list}\NormalTok{(chain),}
                \StringTok{"8"}\NormalTok{ =}\StringTok{ }\KeywordTok{gen_list}\NormalTok{(chain)}
\NormalTok{  )}
\NormalTok{  )}
\NormalTok{\}}

\CommentTok{# fit the conditional model}
\NormalTok{m1 <-}\StringTok{ }\KeywordTok{run.jags}\NormalTok{(}
  \StringTok{"./jags_script/conditional_model_single_season.R"}\NormalTok{,}
  \DataTypeTok{monitor =} \KeywordTok{c}\NormalTok{(}\StringTok{"psi"}\NormalTok{, }\StringTok{"omega"}\NormalTok{,}\StringTok{"rho"}\NormalTok{,}\StringTok{"gamma"}\NormalTok{, }\StringTok{"n_coyote"}\NormalTok{, }\StringTok{"n_mange"}\NormalTok{),}
  \DataTypeTok{data =}\NormalTok{ data_list,}
  \DataTypeTok{n.chains =} \DecValTok{4}\NormalTok{,}
  \DataTypeTok{inits =}\NormalTok{ inits_simulated,}
  \DataTypeTok{burnin =} \DecValTok{10000}\NormalTok{,}
  \DataTypeTok{adapt =} \DecValTok{10000}\NormalTok{,}
  \DataTypeTok{sample =} \DecValTok{20000}\NormalTok{,}
  \DataTypeTok{modules =} \StringTok{'glm'}\NormalTok{,}
  \DataTypeTok{method =} \StringTok{'parallel'}
\NormalTok{)}
\end{Highlighting}
\end{Shaded}

And here are the posterior estimates compared to the simulated
parameters, which are the black dots.

\begin{Shaded}
\begin{Highlighting}[]
\NormalTok{m1 <-}\StringTok{ }\KeywordTok{readRDS}\NormalTok{(}\StringTok{"simulated_posterior.RDS"}\NormalTok{)}
\CommentTok{# put together a plot of the parameter estimates}
\CommentTok{#  and compare them to the true values}
\NormalTok{mcmcplots}\OperatorTok{::}\KeywordTok{caterplot}\NormalTok{(}
\NormalTok{  m1,}
  \DataTypeTok{regex =} \StringTok{"psi|rho|gam|ome"}\NormalTok{,}
  \DataTypeTok{reorder =} \OtherTok{FALSE}
\NormalTok{)}
\end{Highlighting}
\end{Shaded}

\begin{verbatim}
## Warning in as.mcmc.runjags(x): Combining the 4 mcmc chains together
\end{verbatim}

\begin{Shaded}
\begin{Highlighting}[]
\NormalTok{my_pars <-}\StringTok{ }\KeywordTok{c}\NormalTok{(}\FloatTok{0.5}\NormalTok{, }\DecValTok{-1}\NormalTok{, }\FloatTok{-0.5}\NormalTok{, }\FloatTok{0.5}\NormalTok{, }\FloatTok{-0.5}\NormalTok{, }\FloatTok{0.5}\NormalTok{, }\FloatTok{-0.5}\NormalTok{, }\FloatTok{0.25}\NormalTok{, }\FloatTok{0.7}\NormalTok{)}
\KeywordTok{points}\NormalTok{(}
  \KeywordTok{rev}\NormalTok{(}\DecValTok{1}\OperatorTok{:}\DecValTok{9}\NormalTok{) }\OperatorTok{~}\StringTok{ }\NormalTok{my_pars,}
  \DataTypeTok{pch =} \DecValTok{19}
\NormalTok{)}
\end{Highlighting}
\end{Shaded}

\includegraphics{model_simulation_example_files/figure-latex/mcmc_plots-1.pdf}

\end{document}
